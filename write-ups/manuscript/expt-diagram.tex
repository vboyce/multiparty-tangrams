% A Bloch sphere of radius |a| = 1 contains all possible states of a two-state quantum system (qubit).
% Each Bloch vector fully determines a spin-1/2 density matrix.
% Used in Exercise Sheet 10 of Statistical Physics by Manfred Salmhofer (2016), available at https://janosh.dev/physics/statistical-physics.

\documentclass{article}

\usepackage{tikz}

\usetikzlibrary{angles, quotes}
	\usetikzlibrary{arrows.meta, positioning, quotes, shapes, shapes.geometric}
\begin{document}
	
%	\definecolor{baseline6}{HTML}{440154}
%		\definecolor{baseline5}{HTML}{3B528B}
%			\definecolor{baseline4}{HTML}{21908C}
%	\definecolor{baseline3}{HTML}{5DC863}
%		\definecolor{baseline2}{HTML}{FDE725}
%		
%		\definecolor{thick6}{HTML}{1F78B4}
%		\definecolor{thick2}{HTML}{33A02C}
%						\definecolor{thin6}{HTML}{A6CEE3}
%								\definecolor{thin2}{HTML}{B2DF8A}

	\definecolor{baseline6}{HTML}{9000FF}
		\definecolor{baseline5}{HTML}{A12EFF}
			\definecolor{baseline4}{HTML}{D24AFF}
	\definecolor{baseline3}{HTML}{FF7DF0}
		\definecolor{baseline2}{HTML}{FFBDD4}
		
		\definecolor{thick6}{HTML}{00BDA8}
		\definecolor{thick2}{HTML}{77F3DB}
		
	\definecolor{thin6}{HTML}{D47E04}
	\definecolor{thin2}{HTML}{FFDA09}
	
	\definecolor{full}{HTML}{6940FF}
	\definecolor{single}{HTML}{00A2FF}
	
	\definecolor{expt1col}{HTML}{FF0099}
		\definecolor{expt3col}{HTML}{527319}
				\definecolor{expt2col}{HTML}{0000FF}

\tikzset{
	expt1/.style={
		draw, circle, color=expt1col, line width=4, align=center, scale=2},
	expt2/.style={
		draw, circle, color=expt2col, line width=4,  align=center, scale=2},
	expt3/.style={
		draw, circle, color=expt3col, line width=4, align=center, scale=2}
}
\tikzset{nodes={font=\sffamily\bfseries}} 


	\begin{tikzpicture}
	
	% Define radius
	\def\r{6}
	% Axes
	\draw (0,0) edge[-Stealth, line width=1] ["Backchannel", sloped, pos=.3] (-2*\r/5,  -2*\r/3) ;
	\draw (0,0) edge[-Stealth, line width=1] ["Group Size", pos=.9] (\r, 0) ;
	\draw (0,0) edge[-Stealth, line width=1] ["Group coherence", sloped, pos=.3] (0, \r);
	
	
	
	
	\node (n2) at (0,0) [expt1, fill=baseline2, label={[expt1col]below:2}] {};
	\node (n3) at (1,0) [expt1, fill=baseline3, label={[expt1col]below:3}]{};
	\node (n4) at (2,0) [expt1, fill=baseline4, label={[expt1col]below:4}] {};
	\node (n5) at (3,0) [expt1, fill=baseline5, label={[expt1col]below:5}] {};
	\node (n6) at (4,0) [expt1, fill=baseline6] {};
		\node[align=center,anchor=north,text=expt1col] at (4.6,-.45) {6 baseline};
	
	\node (n2thick) at (0,4) [expt3, fill=thick2, label= {[expt3col]above right:2 thick}] {};
	\node (n6thick) at (4,4) [expt3, fill=thick6, label={[expt3col]right:6 thick}] {};
	
	\node (nsingle) at (4,1.5) [expt2, fill=full, label={[expt2col]right:6 full feedback}] {};
	\node (nfull) at (4,3) [expt2, fill=single, label={[expt2col]right:6 single speaker}] {};
	
	\node (n2thin) at (-3/2,-5/2) [expt3, fill=thin2, label={[expt3col]below right: 2 thin}] {};
	
	\node (n6thin2) at (2.7,-2.3) [expt2, fill=thin6, label={[expt2col] right: 6 thin}] {};
	\node (n6thin) at (2.5,-2.5) [expt3, fill=thin6, label={[expt3col]below right: 6 thin}] {};
	
	
	\node[align=center,anchor=north] (lab1) at (-2.5,2) {Rotating speaker;\\limited feedback};
	\node[align=center,anchor=north] (lab2) at (-2.5,4) {Single speaker;\\full feedback};
	\node[align=center,anchor=north] (lab3) at (-2,.5) {Listeners \\use chat};
	\node[align=center,anchor=north] (lab4) at (-3.5,-2) {Listeners \\use emoji};
	
	\node[align=center,anchor=north, text=expt1col] (ex1) at (2,1) {Experiment 1};
	\node[align=center,anchor=north, text=expt2col] (ex1) at (2,2.75) {Experiment 2};
	\node[align=center,anchor=north, text=expt3col] (ex1) at (.7,-1.7) {Experiment 3};
	
	\draw[dashed] (n6) edge [] (n6thin);
	\draw[dashed] (n2thin) edge [] (n6thin);
	\draw[dashed] (n2thick) edge [] (n6thick);
	\draw[dashed] (n6) edge [] (n6thick);
	
	\begin{scope}[line width=1 pt, >=Stealth]
		\draw[->] (lab1) -> (n2);
		\draw[->] (lab2) -> (n2thick);
		\draw[->] (lab3) -> (n2);
		\draw[->] (lab4) -> (n2thin);
		
	\end{scope}
	
\end{tikzpicture}

\end{document}
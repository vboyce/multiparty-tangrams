% A Bloch sphere of radius |a| = 1 contains all possible states of a two-state quantum system (qubit).
% Each Bloch vector fully determines a spin-1/2 density matrix.
% Used in Exercise Sheet 10 of Statistical Physics by Manfred Salmhofer (2016), available at https://janosh.dev/physics/statistical-physics.

\documentclass{article}

\usepackage{tikz}

\usetikzlibrary{angles, quotes}
	\usetikzlibrary{arrows.meta, positioning, quotes, shapes, shapes.geometric}
\begin{document}
	
	\definecolor{baseline6}{HTML}{440154}
		\definecolor{baseline5}{HTML}{3B528B}
			\definecolor{baseline4}{HTML}{21908C}
	\definecolor{baseline3}{HTML}{5DC863}
		\definecolor{baseline2}{HTML}{FDE725}
		
		\definecolor{thick6}{HTML}{1F78B4}
		\definecolor{thick2}{HTML}{33A02C}
						\definecolor{thin6}{HTML}{A6CEE3}
								\definecolor{thin2}{HTML}{B2DF8A}
\tikzset{
	expt1/.style={
		draw, circle,  align=center, scale=1},
	expt2/.style={
		draw, rectangle,  align=center, scale=1},
	expt3/.style={
		draw, isosceles triangle, isosceles triangle apex angle=60,  rotate=90, scale=.7}
}


	\begin{tikzpicture}
	
	% Define radius
	\def\r{6}
	% Axes
	\draw (0,0) edge[-Stealth, line width=1] ["Backchannel", sloped, pos=.3] (-2*\r/5,  -2*\r/3) ;
	\draw (0,0) edge[-Stealth, line width=1] ["Group Size", pos=.9] (\r, 0) ;
	\draw (0,0) edge[-Stealth, line width=1] ["Group coherence", sloped, pos=.3] (0, \r);
	
	
	
	
	\node (n2) at (0,0) [expt1, fill=baseline2, label=above right:2 baseline] {};
	\node (n3) at (1,0) [expt1, fill=baseline3, label=below:3] {};
	\node (n4) at (2,0) [expt1, fill=baseline4, label=below:4] {};
	\node (n5) at (3,0) [expt1, fill=baseline5, label=below:5] {};
	\node (n6) at (4,0) [expt1, fill=baseline6, label=below right:6 baseline] {};
	
	\node (n2thick) at (0,4) [expt3, fill=thick2, label=below right:2 thick] {};
	\node (n6thick) at (4,4) [expt3, fill=thick6, label=below right:6 thick] {};
	
	\node (nsingle) at (4,1.5) [expt2, fill=orange, label=right:6 single speaker] {};
	\node (nfull) at (4,3) [expt2, fill=pink, label=right:6 full feedback] {};
	
	\node (n2thin) at (-3/2,-5/2) [expt3, fill=thin2, label=below left: 2 thin] {};
	
	\node (n6thin2) at (2.4,-2.3) [expt2, fill=red] {};
	\node (n6thin) at (2.5,-5/2) [expt3, fill=thin6, label=below left: 6 thin] {};
	
	
	\node[align=center,anchor=north] (lab1) at (-2,2) {Rotating speaker;\\limited feedback};
	\node[align=center,anchor=north] (lab2) at (-2,6) {Single speaker;\\full feedback};
	\node[align=center,anchor=north] (lab3) at (-2,0) {Listeners \\use chat};
	\node[align=center,anchor=north] (lab4) at (-4,-2) {Listeners \\use emoji};
	
	
	\draw[dashed] (n6) edge [] (n6thin);
	\draw[dashed] (n2thin) edge [] (n6thin);
	\draw[dashed] (n2thick) edge [] (n6thick);
	\draw[dashed] (n6) edge [] (n6thick);
	
	\begin{scope}[line width=1 pt, >=Stealth]
		\draw[->] (lab1) -> (n2);
		\draw[->] (lab2) -> (n2thick);
		\draw[->] (lab3) -> (n2);
		\draw[->] (lab4) -> (n2thin);
		
	\end{scope}
	
\end{tikzpicture}

\tikzset{
		expt1/.style={
		draw, circle, fill=blue, align=center, scale=1},
	expt2/.style={
		draw, circle, fill=red, align=center, scale=1},
	expt3/.style={
		draw, circle, fill=green, rotate=90, scale=1}
}
	\begin{tikzpicture}
	
	% Define radius
	\def\r{6}
	% Axes
	\draw (0,0) edge[-Stealth, line width=1] ["Backchannel", sloped, pos=.3] (-2*\r/5,  -2*\r/3) ;
	\draw (0,0) edge[-Stealth, line width=1] ["Group Size", pos=.9] (\r, 0) ;
	\draw (0,0) edge[-Stealth, line width=1] ["Group coherence", sloped, pos=.3] (0, \r);
	
	
	
	
	\node (n2) at (0,0) [expt1, label=above right:2 baseline] {};
	\node (n3) at (1,0) [expt1, label=below:3] {};
	\node (n4) at (2,0) [expt1, label=below:4] {};
	\node (n5) at (3,0) [expt1,  label=below:5] {};
	\node (n6) at (4,0) [expt1,  label=below right:6 baseline] {};
	
	\node (n2thick) at (0,4) [expt3,  label=below right:2 thick] {};
	\node (n6thick) at (4,4) [expt3,  label=below right:6 thick] {};
	
	\node (nsingle) at (4,1.5) [expt2,  label=right:6 single speaker] {};
	\node (nfull) at (4,3) [expt2,  label=right:6 full feedback] {};
	
	\node (n2thin) at (-3/2,-5/2) [expt3,  label=below left: 2 thin] {};
	
	\node (n6thin2) at (2.4,-2.3) [expt2, ] {};
	\node (n6thin) at (2.5,-5/2) [expt3,  label=below left: 6 thin] {};
	
	
	\node[align=center,anchor=north] (lab1) at (-2,2) {Rotating speaker;\\limited feedback};
	\node[align=center,anchor=north] (lab2) at (-2,6) {Single speaker;\\full feedback};
	\node[align=center,anchor=north] (lab3) at (-2,0) {Listeners \\use chat};
	\node[align=center,anchor=north] (lab4) at (-4,-2) {Listeners \\use emoji};
	
	
	\draw[dashed] (n6) edge [] (n6thin);
	\draw[dashed] (n2thin) edge [] (n6thin);
	\draw[dashed] (n2thick) edge [] (n6thick);
	\draw[dashed] (n6) edge [] (n6thick);
	
	\begin{scope}[line width=1 pt, >=Stealth]
		\draw[->] (lab1) -> (n2);
		\draw[->] (lab2) -> (n2thick);
		\draw[->] (lab3) -> (n2);
		\draw[->] (lab4) -> (n2thin);
		
	\end{scope}
	
\end{tikzpicture}

Thinking we can color coordinate dots to match what we do in graphs (unclear on what the color scheme should be...)

Currently using shape for the 3 expts

Not sure how to label the expt 1's or the game size axis

Shape is expt number for now

2c and 6thin being the same but probably shown on different graphs is awkward

\end{document}
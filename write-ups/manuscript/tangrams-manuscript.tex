% Options for packages loaded elsewhere
\PassOptionsToPackage{unicode}{hyperref}
\PassOptionsToPackage{hyphens}{url}
\PassOptionsToPackage{dvipsnames,svgnames,x11names}{xcolor}
%
\documentclass[
  english,
  a4paper,
]{article}
\usepackage{amsmath,amssymb}
\usepackage{lmodern}
\usepackage{iftex}
\ifPDFTeX
  \usepackage[T1]{fontenc}
  \usepackage[utf8]{inputenc}
  \usepackage{textcomp} % provide euro and other symbols
\else % if luatex or xetex
  \usepackage{unicode-math}
  \defaultfontfeatures{Scale=MatchLowercase}
  \defaultfontfeatures[\rmfamily]{Ligatures=TeX,Scale=1}
\fi
% Use upquote if available, for straight quotes in verbatim environments
\IfFileExists{upquote.sty}{\usepackage{upquote}}{}
\IfFileExists{microtype.sty}{% use microtype if available
  \usepackage[]{microtype}
  \UseMicrotypeSet[protrusion]{basicmath} % disable protrusion for tt fonts
}{}
\makeatletter
\@ifundefined{KOMAClassName}{% if non-KOMA class
  \IfFileExists{parskip.sty}{%
    \usepackage{parskip}
  }{% else
    \setlength{\parindent}{0pt}
    \setlength{\parskip}{6pt plus 2pt minus 1pt}}
}{% if KOMA class
  \KOMAoptions{parskip=half}}
\makeatother
\usepackage{xcolor}
\IfFileExists{xurl.sty}{\usepackage{xurl}}{} % add URL line breaks if available
\IfFileExists{bookmark.sty}{\usepackage{bookmark}}{\usepackage{hyperref}}
\hypersetup{
  pdftitle={Interaction structure constrains the emergence of conventions in group communication},
  pdflang={en},
  colorlinks=true,
  linkcolor={blue},
  filecolor={Maroon},
  citecolor={blue},
  urlcolor={blue},
  pdfcreator={LaTeX via pandoc}}
\urlstyle{same} % disable monospaced font for URLs
\usepackage[margin=25mm]{geometry}
\usepackage{longtable,booktabs,array}
\usepackage{calc} % for calculating minipage widths
% Correct order of tables after \paragraph or \subparagraph
\usepackage{etoolbox}
\makeatletter
\patchcmd\longtable{\par}{\if@noskipsec\mbox{}\fi\par}{}{}
\makeatother
% Allow footnotes in longtable head/foot
\usepackage{footnote} % For some unknown reason, footnotehyper clashes with French
\makesavenoteenv{longtable}
\usepackage{graphicx}
\makeatletter
\def\maxwidth{\ifdim\Gin@nat@width>\linewidth\linewidth\else\Gin@nat@width\fi}
\def\maxheight{\ifdim\Gin@nat@height>\textheight\textheight\else\Gin@nat@height\fi}
\makeatother
% Scale images if necessary, so that they will not overflow the page
% margins by default, and it is still possible to overwrite the defaults
% using explicit options in \includegraphics[width, height, ...]{}
\setkeys{Gin}{width=\maxwidth,height=\maxheight,keepaspectratio}
% Set default figure placement to htbp
\makeatletter
\def\fps@figure{htbp}
\makeatother
\setlength{\emergencystretch}{3em} % prevent overfull lines
\providecommand{\tightlist}{%
  \setlength{\itemsep}{0pt}\setlength{\parskip}{0pt}}
\setcounter{secnumdepth}{5}
\newlength{\cslhangindent}
\setlength{\cslhangindent}{1.5em}
\newlength{\csllabelwidth}
\setlength{\csllabelwidth}{3em}
\newlength{\cslentryspacingunit} % times entry-spacing
\setlength{\cslentryspacingunit}{\parskip}
\newenvironment{CSLReferences}[2] % #1 hanging-ident, #2 entry spacing
 {% dont indent paragraphs
  \setlength{\parindent}{0pt}
  % turn on hanging indent if param 1 is 1
  \ifodd #1
  \let\oldpar\par
  \def\par{\hangindent=\cslhangindent\oldpar}
  \fi
  % set entry spacing
  \setlength{\parskip}{#2\cslentryspacingunit}
 }%
 {}
\usepackage{calc}
\newcommand{\CSLBlock}[1]{#1\hfill\break}
\newcommand{\CSLLeftMargin}[1]{\parbox[t]{\csllabelwidth}{#1}}
\newcommand{\CSLRightInline}[1]{\parbox[t]{\linewidth - \csllabelwidth}{#1}\break}
\newcommand{\CSLIndent}[1]{\hspace{\cslhangindent}#1}

%%%%%%%% START HEADER PARTIAL %%%%%%%%%%%%

% Formatting of tables & knitr::kable and kableExtra functionality
\usepackage{float}
\usepackage{colortbl}
\usepackage{pdflscape}
\usepackage{tabu}
\usepackage{threeparttable}

% Line numbering

% endfloat stuff

% fancyhdr pagestyle

% Environment for keywords
\makeatletter
\newcommand\keywordsname{Keywords}
\newenvironment*{keywords}[1][\keywordsname]{\if@twocolumn \else \small \quotation \fi \begin{center} \textbf{\textit{#1} \\}}{\end{center}\if@twocolumn \else \small \endquotation \fi}
\newenvironment*{keywordsinline}[1][\keywordsname]{\if@twocolumn \else \small \quotation \fi \begin{center} \textbf{\textit{#1}: }}{\end{center}\if@twocolumn \else \small \endquotation \fi}
\makeatother

% Environment for abstract that takes new abstract name
\newenvironment{renameableabstract}[1][\abstractname]{\let\oldabstractname\abstractname \renewcommand{\abstractname}{#1} \begin{abstract}}{\end{abstract} \renewcommand{\abstractname}{\oldabstractname}}

%%%%%%%% END HEADER PARTIAL %%%%%%%%%%%%

\usepackage{tikz}
\usetikzlibrary{angles,positioning,arrows.meta, quotes, shapes, shapes.geometric}
\usepackage{graphicx}
\usepackage{booktabs}
\usepackage{longtable}
\usepackage{array}
\usepackage{multirow}
\usepackage{wrapfig}
\usepackage{float}
\usepackage{colortbl}
\usepackage{pdflscape}
\usepackage{tabu}
\usepackage{threeparttable}
\usepackage{threeparttablex}
\usepackage[normalem]{ulem}
\usepackage{makecell}
\usepackage{xcolor}
\ifXeTeX
  % Load polyglossia as late as possible: uses bidi with RTL langages (e.g. Hebrew, Arabic)
  \usepackage{polyglossia}
  \setmainlanguage[]{}
\else
  \usepackage[english,main=english]{babel}
% get rid of language-specific shorthands (see #6817):
\let\LanguageShortHands\languageshorthands
\def\languageshorthands#1{}
\fi
\ifLuaTeX
  \usepackage{selnolig}  % disable illegal ligatures
\fi

\title{Interaction structure constrains the emergence of conventions in group communication}

%%%%%%% START AUTHOR PARTIAL %%%%%%%%%%%%%%%

%%%%% Authors, affiliations and author notes stuff %%%%%

% Macros for creating and referencing stored reference
\makeatletter
\def\MyNewLabel#1#2#3{\expandafter\gdef\csname #1@#2\endcsname{#3}}

\def\MyRef#1#2{\@ifundefined{#1@#2}{???}{\csname #1@#2\endcsname}}

\newcommand*\ifcounter[1]{%
  \ifcsname c@#1\endcsname
    \expandafter\@firstoftwo
  \else
    \expandafter\@secondoftwo
  \fi
}
\makeatother

% Create labels for Addresses if the are given by code
\MyNewLabel{ADDRTXT}{Stanford}{Stanford University}
\MyNewLabel{ADDRTXT}{Princeton}{Princeton University}

% Create labels for Footnotes if they are given by code
\MyNewLabel{ANOTETXT}{corresp}{Corresponding author. Email: \href{mailto:vboyce@stanford.edu}{\nolinkurl{vboyce@stanford.edu}}}

%%% Special footnotes for addresses and author footnotes
\usepackage{bigfoot}
\DeclareNewFootnote{Addr}[arabic] % Only used for NOT authblk
\DeclareNewFootnote{ANote}[fnsymbol]

%%% Address and author notes as a function of format %%%
 % Use authblk for affiliations %%%%%%%%%%%
\usepackage{authblk}

% Always separate by commas
\renewcommand\Authsep{, }
\renewcommand\Authand{, }
\renewcommand\Authands{, }

% Counter for addresses and footnotes
\newcounter{addrcnt}

% thanks definition that doesnt produce superscript marks
\makeatletter
\newcommand*\createaddrlblbycode[1]{%
  \ifcounter{ADDRLBL@#1}
    {}
    {\refstepcounter{addrcnt}\newcounter{ADDRLBL@#1}\setcounter{ADDRLBL@#1}{\value{addrcnt}}}%
}

\newcommand*\addrlblbycode[1]{\arabic{ADDRLBL@#1}}

\newcommand*\addrbycode[1]{%
  \ifcounter{ADDR@#1}
    {}
    {\newcounter{ADDR@#1}%
     \affil[\addrlblbycode{#1}]{\MyRef{ADDRTXT}{#1}}}%
}

\newcommand*\createanotelblbycode[1]{%
  \ifcounter{ANOTELBL@#1}
    {}
    {\refstepcounter{footnoteANote}\newcounter{ANOTELBL@#1}\setcounter{ANOTELBL@#1}{\value{footnoteANote}}}%
}

\newcommand*\anotelblbycode[1]{\fnsymbol{ANOTELBL@#1}}

\newcommand*\anotebycode[1]{%
  \ifcounter{ANOTE@#1}
    {}
    {\newcounter{ANOTE@#1}%
     \footnotetextANote[\value{ANOTELBL@#1}]{\MyRef{ANOTETXT}{#1}}}%
}
\makeatother


\createaddrlblbycode{Stanford}


\createanotelblbycode{corresp}

\author[%
\addrlblbycode{Stanford}%
,%
$\anotelblbycode{corresp}$%
]{Veronica Boyce}

\addrbycode{Stanford}


\createaddrlblbycode{Princeton}



\author[%
\addrlblbycode{Princeton}%
]{Robert Hawkins}

\addrbycode{Princeton}


\createaddrlblbycode{Stanford}



\author[%
\addrlblbycode{Stanford}%
]{Noah D. Goodman}

\addrbycode{Stanford}


\createaddrlblbycode{Stanford}



\author[%
\addrlblbycode{Stanford}%
]{Michael C. Frank}

\addrbycode{Stanford}


%endif(authblk)

%%%%%%%%% END AUTHOR PARTIAL %%%%%%%%

\date{}

\begin{document}
\maketitle

%%%%%%%%%% START AFTER TITLE PARTIAL %%%%%%%%%%%%%
\anotebycode{corresp}


%%%%%%%%%% END AFTER TITLE PARTIAL %%%%%%%%%%%%%


\begin{otherlanguage}{english}

\begin{abstract}
Group communication is ubiqutous, but it presents some challenges. however, potential challenges from larger groups aren't studied much because the literature is primarily dyadic. One key phenomenon, reduction over repeated reference, is well-attested and could explain how pairs build shared meaning.
We extend the same paradigm across a NUMBER of games varying in group size and interaction structure. We find across conditions that reduction and convergence to shared descriptions occurs, and there is a gradient depending both on group size and other features of group interaction.

\end{abstract}

\end{otherlanguage}

Communicating in groups can be challenging. People may each have different background knowledge on the topic of conversation, and may not know who knows what. Everyone may have different questions or different elements to focus on which can result in everyone talking at once, and thus confusion. The group might splinter into side conversations, which creates more unshared knowledge that may eventually need to be merged if the group needs to reach agreement or consensus. Disagreements over the issues or over scope or even over what is being referred to can interfere with efficient communication. If the level of interrupting and simultaneous talking gets to be too much, there may even be meta-disagreements about how the conversation is structured. Not all of these impediments occur in every group conversation, but we've all been in situations where having a conversation with a half-dozen people at once can devolve into chaos and be an inefficient way of communicating. {[}TODO what are the key empathy-inducing problems of group convo and how can we list them succinctly{]}

Despite all of these factors, we often do manage to communicate with groups.

How?

{[}work on this transition{]}

\begin{itemize}
\tightlist
\item
  one key thing that convos need to be successful is BLAH
\item
  can be studied in more isolated forms so is used as a case study
\item
  Dyadic work shows \ldots.
\item
  But work has been mostly dyadic -- which misses some of the complexities of multi-party communication that could influence this. --\textgreater{} How do groups perform this task and what can we learn about mutli-party from that?
\end{itemize}

As alluded to above, groups present some dynamics and challenges not present in dyadic contexts, but most of the work on communication has used dyadic contexts.
These multi-party conversational settings are a common feature of our lives as social beings, but most of the work trying to understand communication, especially referential communication, is studied in dyadic contexts. One key requirement for efficient communication requires some sort of shared labels for the topics of conversation (\protect\hyperlink{ref-traum2004}{Traum 2004}, \protect\hyperlink{ref-ginzburg2005}{Ginzburg \& Fernandez 2005}, \protect\hyperlink{ref-branigan2006}{Branigan 2006}). In many cases, there are widely shared conventionalized expressions for objects or ideas, but in other cases, spontaneous ad-hoc expressions must be invented.

The formation of these new reference expressions is well-studied in dyadic contexts and has been a case study for efficient communication more broadly. Clark \& Wilkes-Gibbs (\protect\hyperlink{ref-clarkReferringCollaborativeProcess1986}{1986}) established an experimental method for studying the emergence of new referring expressions that has now become standard (building on \protect\hyperlink{ref-kraussChangesReferencePhrases1964}{Krauss \& Weinheimer 1964}, \protect\hyperlink{ref-kraussConcurrentFeedbackConfirmation1966}{1966}). Two participants see the same set of figures; the speaker describes each figure in turn so the listener can select the target from the set of figures. The speaker and listener repeat this process with the same images over a series of blocks. Early descriptions are long and make reference to multiple features in the figure, but in later iterations, shorthand conventional names for each figure emerge; this shortening of utterances is called `reduction'.

Recently, online participant recruitment and web-based experiments have made it possible to study this convergence in larger populations (\protect\hyperlink{ref-haber2019}{Haber et al. 2019}, \protect\hyperlink{ref-hawkinsCharacterizingDynamicsLearning2020}{Hawkins et al. 2020}). In line with results from face-to-face, oral paradigms, speakers reduced their utterances, producing fewer words per image in later blocks than in earlier blocks. (Throughout this paper, we use ``speaker'' and ``listener'' to refer to the roles describing and selecting targets, regardless of communication modality.)

Dyads are robustly successful at forming conventionalized shorthands in iterated reference games, but talking with one person is simpler than talking with several. How does the formation of shared conventions proceed in multi-party communication? In a dyad, speakers can tailor their utterances to the one listener, but in large groups, speakers must balance the competing needs of different listeners (\protect\hyperlink{ref-schober1989}{Schober \& Clark 1989}, \protect\hyperlink{ref-tolins2016}{Tolins \& Fox Tree 2016}). These effects likely vary by both the knowledge state of and communication channels available to the listeners (\protect\hyperlink{ref-horton2002}{Horton \& Gerrig 2002}, \protect\hyperlink{ref-horton2005}{Horton \& Gerrig 2005}, \protect\hyperlink{ref-fox-tree2013}{Fox Tree \& Clark 2013}).

MENTION THAT THIS IS ONLINE

In the current work, we address how components of interaction structure, including group size and communication channels, shape how successfully groups form partner-specific conventionalized names for target objects over the course of an iterated reference game. We recruited 1319 participants who were organized into 313 groups distributed across 3 experiments and 11 conditions. Collectively, players produced 326000 during their games. We analysed the results using traditional metrics of accuracy and number of words, and we used computation measures of semantic similarity to understand how utterances evolve during the games.

We find that the characteristic pattern of increasing accuracy and decreasing utterance length that is noted in the dyadic literature also occurs across conditions in multi-player games. Reduction is coupled with semantic shifts as utterances converge toward the eventual conventionalized name and away from descriptions used by other groups or for other tangrams. These convention-formation phenomena emerge across disparate conditions; however, there are gradient effects where larger groups and groups with narrower communication channels are less effective and converge more slowly and weakly than other groups.

\definecolor{baseline6}{HTML}{6940FF}
\definecolor{baseline5}{HTML}{A12EFF}
\definecolor{baseline4}{HTML}{D24AFF}
\definecolor{baseline3}{HTML}{FF7DF0}
\definecolor{baseline2}{HTML}{FFBDD4}

\definecolor{thick6}{HTML}{00BDA8}
\definecolor{thick2}{HTML}{77F3DB}

\definecolor{thin6}{HTML}{D47E04}
\definecolor{thin2}{HTML}{FFDA09}

\definecolor{single}{HTML}{00A2FF}
\definecolor{full}{HTML}{425df5}

\definecolor{expt1col}{HTML}{FF0099}
\definecolor{expt3col}{HTML}{527319}
\definecolor{expt2col}{HTML}{0000FF}

\tikzset{
    expt1/.style={
        draw, circle, color=expt1col, line width=4, align=center, scale=2},
    expt2/.style={
        draw, circle, color=expt2col, line width=4,  align=center, scale=2},
    expt3/.style={
        draw, circle, color=expt3col, line width=4, align=center, scale=2},
        illus/.style={draw, rectangle}
}

\tikzset{nodes={font=\sffamily\bfseries}} 

\begin{figure}
    
    
    
    \begin{tikzpicture}
        
        \def\r{5}
        \draw (0,0) edge[-Stealth, line width=1] ["Backchannel", sloped, pos=.4] (-2*\r/5,  -2*\r/3) ;
        \draw (0,0) edge[-Stealth, line width=1] ["Group Size", pos=1.05] (\r, 0) ;
        \draw (0,0) edge[-Stealth, line width=1] ["Group coherence", sloped, pos=.4] (0, \r);
        
        \node (n2) at (0,0) [expt1, fill=baseline2, label={[expt1col]below:2}] {};
        \node (n3) at (1,0) [expt1, fill=baseline3, label={[expt1col]below:3}]{};
        \node (n4) at (2,0) [expt1, fill=baseline4, label={[expt1col]below:4}] {};
        \node (n5) at (3,0) [expt1, fill=baseline5, label={[expt1col]below:5}] {};
        \node (n6) at (4,0) [expt1, fill=baseline6] {};
        \node[align=center,anchor=north,text=expt1col] at (4.6,-.45) {6 baseline};
        
        \node (n2thick) at (0,4) [expt3, fill=thick2, label= {[expt3col]above right:2 thick}] {};
        \node (n6thick) at (4,4) [expt3, fill=thick6, label={[expt3col]right:6 thick}] {};
        
        \node (nfull) at (4,1.5) [expt2, fill=full, label={[expt2col]right:6 full feedback}] {};
        \node (nsingle) at (4,3) [expt2, fill=single, label={[expt2col]right:6 single speaker}] {};
        
        \node (n2thin) at (-3/2,-5/2) [expt3, fill=thin2, label={[expt3col]below right: 2 thin}] {};
        \node (n6thin2) at (2.7,-2.3) [expt2, fill=thin6, label={[expt2col] right: 6 thin}] {};
        \node (n6thin) at (2.5,-2.5) [expt3, fill=thin6, label={[expt3col]below right: 6 thin}] {};
        
        \node[align=center,anchor=north] (lab1) at (-2,3) {Rotating speaker;\\limited feedback};
        \node[align=center,anchor=north] (lab2) at (-1.5,6) {Single speaker;\\full feedback};
        \node[align=center,anchor=north] (lab3) at (-3,0.2) {Listeners \\use chat};
        \node[align=center,anchor=north] (lab4) at (-3.5,-2) {Listeners \\use emoji};
        
        \node[align=center,anchor=north, text=expt1col] (ex1) at (2,1) {Experiment 1};
        \node[align=center,anchor=north, text=expt2col] (ex1) at (2,2.75) {Experiment 2};
        \node[align=center,anchor=north, text=expt3col] (ex1) at (.7,-1.7) {Experiment 3};
        
     \node[illus, anchor=north] () at (-3.5,-3.5) {  \includegraphics[height=2.5em]{images/nochat.png}};
    \node[illus, anchor=north] () at (-3.3,-.7) {\includegraphics[height=2.5em]{images/chat.png}};
    
    \node[illus, anchor=north] () at (-3.35,2) {\includegraphics[height=3em]{images/rotate.png}};
    \node[illus, anchor=north] () at (7.5,4.8) {\includegraphics[height=3em]{images/norotate.png}};
    
    
    \node[anchor=north] () at (-1,.7) {\includegraphics[height=1.5em]{images/two.png}};
    \node[ anchor=north] () at (6.8,.8) {\includegraphics[height=2em]{images/six.png}};
    
    \node[illus, anchor=north] () at (-4.3,4) { \includegraphics[height=4em]{images/limited-wrong.png}    %\includegraphics[scale=.1]{images/limited-correct.png}
    };
    \node[illus, anchor=north] () at (8,2.5) {  \includegraphics[height=4em]{images/full-wrong.png}   %\includegraphics[scale=.1]{images/full-correct.png}
        };
        
        \draw[dashed] (n6) edge [] (n6thin);
        \draw[dashed] (n2thin) edge [] (n6thin);
        \draw[dashed] (n2thick) edge [] (n6thick);
        \draw[dashed] (n6) edge [] (n6thick);
        
        
        \node[illus, anchor=north west] () at (4.75, -1.5) {\includegraphics[width=15em]{images/diagram.png}};
        \node[anchor=north west] () at (7, -1) {Trial};
        
        \node[illus, anchor=north west, minimum width=18.5em, minimum height=14.5em] (block) at (4.5, -1) {};
        \node[anchor=north west] () at (7, -.5) {Block};
        \node[anchor=north west] () at (8.5, -5.5) {x 12 images};
        \node[anchor=north west] () at (8.5, -6.1) {x 6 iterations};
        
        \begin{scope}[line width=1 pt, >=Stealth]
            \draw[->] (lab1) -> (n2);
            \draw[->] (lab2) -> (n2thick);
            \draw[->] (lab3) -> (n2);
            \draw[->] (lab4) -> (n2thin);
            
        \end{scope}
        
        
        
    \end{tikzpicture}
    
\caption{ Diagram of the experimental space explored in these experiments. Experiment 1 (pink) has a backchannel where listeners can use the chat and low group coherence from a rotating speaker and limited feedback. Experiment 1's conditions vary the group size from 2 - 6 players. Experiment 2 (blue) games keep group size constant at 6 and vary along the other dimensions. 6 single speaker and 6 full feedback each add one component of group coherence relative to experiment 1. 6 thin varies the backchannel (relative to experiment 1) by having listeners communicate with emoji rather than the full text chat the speaker uses. Experiment 3 (green) tests 4 corners of the space, crossing group size (2 or 6 players) with thin games that have low group coherence and low backchannel or thick games that have high group coherence and high backchannel.  The not-yet-an-inset shows the structure of the experiment, each trial a speaker describes a target image to the listeners, and this process repeats for all 12 images to comprise a block, and the block repeats for a total of 6 iterations. TODO how to make the inset look inset and what to do with the diagram bits?! }
    \label{diagram}
\end{figure}

\hypertarget{results}{%
\section{Results}\label{results}}

{[}mini-methods{]}

We extended on the dyadic paradigm of Hawkins et al. (\protect\hyperlink{ref-hawkinsCharacterizingDynamicsLearning2020}{2020}) by parameterizing the experiments along a few dimensions while keeping other aspects of the experiment constant. As shown in Figure \ref{diagram} inset, all of the games used the same 12 target images (\protect\hyperlink{ref-clarkReferringCollaborativeProcess1986}{Clark \& Wilkes-Gibbs 1986}, \protect\hyperlink{ref-hawkinsCharacterizingDynamicsLearning2020}{Hawkins et al. 2020}). The speaker knew which image was the target, and their goal was to describe it to the listeners over a chat interface so each listener could select the target. After all listeners had selected, players received feedback on the selections. The process repeated with the same speaker describing each of the 12 images to form one block. The games consisted of 6 blocks, for a total of 72 trials, where each image was described 6 times over the course of the game.

Figure \ref{diagram} schematically illustrates the dimensions of variation between games and where each condition fit in the experimental space. Game size (shown on the x-axis) which varied between 2 and 6 players groups to explore the gradual effects having a larger audience. Group coherence (y-axis) was made up of two components: speaker rotation and feedback. In low group coherence games, the speaker rotated each block, while in high group coherence games, one player was the speaker for the entire game. Rotating speakers is a more stringent test of convention formation because it compares utterances from different players, but having a single speaker adds continuity that can help hold a group together. In low group coherence games, each listener only received feedback on if they were individually right or wrong in their selection; while in high group coherence games, listeners (like the speakers in all games) saw who had selected what and what the target had been, thus ensuring people saw what referent was intended and had a sense of how everyone else was doing. Listener backchannel (z-axis) varied how listeners could communicate with the group. In high backchannel games, the listeners could type text messages to the shared chat; while in low backchannel games, listeners could send 4 discrete messages (represented as emojis) to the chat. This dimension was inspired by the claimed importance of listener contributions to convention formation {[}TODO CIATIONS{]}.

Experiment 1 varied group size with games of 2,3,4,5 or 6 players all with low group coherence and high listener backchannel. Experiment 2 held group size constant at 6 while each condition deviated from experiment 1 in one aspect: 6 single speaker and 6 full feedback changed components of group coherence and 6 thin switched to a low backchannel. Finally, experiment 3 tested 4 corners of the experimental space at larger scale, with thin (low backchannel, low coherence) and thick (high backchannel, high coherence) games with either 2 or 6 players.

We first compare across all of these conditions on the two behavioral measures that are the common markers of reduction in the literature: listener accuracy and speaker reduction {[}CITE{]}. We then explore look at how the speaker's language changes within and between games over time by looking at similarities between utterances.

\hypertarget{behavioral-results}{%
\subsection{Behavioral results}\label{behavioral-results}}

The two key behavioral outcomes were how accurately listeners selected the target images and how many words the speaker produced each trial.

\begin{figure*}[t!]

{\centering \includegraphics[width=1\linewidth]{figs/behavioral-1} 

}

\caption{Behavioral results across all three experiments. A. Listener accuracy at selecting the target image. Dots are per condition, per block estimates with 95\% bootstrapped CIs. Smooths are binomial fit lines. B. Number of words said by the speaker each trial. Faint dots represent individual trials from individual games. Smooths are quadratic fit lines. Y-axis is truncated, and a few outliers points are not visible. }\label{fig:behavioral}
\end{figure*}

In every condition, listeners selected the correct target far above chance, and listener accuracy rose over the course of the game (Figure \ref{fig:behavioral}A). Experiment 1 found no strong effects of group size on overall accuracy (numPlayers: -0.07 {[}-0.2, 0.05{]}) or improvement rate (block:numPlayers: -0.02 {[}-0.05, 0.01{]}). In experiment 3, participants in six player games were less accurate (gameSize6: -0.64 {[}-1.05, -0.25{]}) and slower to improve (block:gameSize6: -0.34 {[}-0.43, -0.25{]}) than players in 2 player games, but thin versus thick games did not have a clear effect on accuracy (channelthin: -0.36 {[}-0.78, 0.05{]}) or improvement rate (block:channelthin: -0.07 {[}-0.18, 0.04{]}). The high and increasing levels of accuracy indicate that across all of these conditions, participants are able to play the game and succeed in communicating about the images.

The key observation in iterated reference games is that the descriptions the speaker gives of the target images become shorter over the course of repetitions. This pattern of reduction held across all conditions, with the numbers of words from the speaker decreasing over blocks (Figure \ref{fig:behavioral}B), although there were substantial differences in how verbose speakers were across games. In experiment 1, the overall effect of being one block later was -3.37 {[}-4.54, -2.24{]} words per trial. Speakers in larger groups said more; the effect of each additional player was 1.66 {[}0.66, 2.61{]} more words per trial, with no clear interaction between block and group size (block:numPlayers: -0.1 {[}-0.36, 0.17{]}). In experiment 2, the result of being one block later was block: -5.39 {[}-6.46, -4.31{]} words per trial for 6 single speaker; block: -4.68 {[}-5.88, -3.52{]} words per for 6 full feedback, and block: -2.15 {[}-3.44, -1.12{]} words per trial for 6 thin. In experiment 3, the six player games said more to start with (gameSize6: 7.41 {[}3.57, 11.18{]}) and reduced less (block:gameSize6: -1.21 {[}-2.06, -0.3{]}) than the two-player games. There were not significant differences due to channel type (channelthin: 0.63 {[}-3.18, 4.73{]}) or channel type over time (block:channelthin: 0.32 {[}-0.65, 1.24{]}).

TODO: dealing with non-centeredness of some of these models! Especially relevant re exp 3 where that are significant differences!

TODO possibly say more about group to group variation

These reduction results confirm and extend what was previously known for 2 player games. Behaviorally, larger games are similar to smaller games, but their speakers tend to say more overall, perhaps related to the increased number of listeners to respond to.

What about the listeners? Compared to how much referential language speakers produce, listeners produce very little, and it is concentrated in the early rounds. TODO say more about listener language including some numbers!!!

\hypertarget{comparisons-of-language-between-and-within-games}{%
\subsection{Comparisons of language between and within games}\label{comparisons-of-language-between-and-within-games}}

\begin{figure}


        \definecolor{div}{HTML}{006400}
\tikzset{
mynode/.style={
draw, rectangle, align=center,  text width=4cm, scale=1, font=\small, inner sep=.5ex, anchor=north east},
}

\begin{tikzpicture}
\node[mynode] (a0) at (1.5, 9.6) {\textbf{Round 1:} stomping feet, head is not the highest point of the picture - maybe a shoulder is at the same height, both feet point left};
\node[mynode] (a1) at (1.5, 7.1)  {\textbf{Round 2:} diamond head is almost on back and on right side of body; both feet are pointing left
};
\node[mynode] (a2) at (1.5, 5.4)  {\textbf{Round 3: }sad guy with square backpack, feet to the left
};
\node[mynode] (a3) at (1.5, 4.1) { \textbf{Round 4:} stomping feet pointing left; square backpack on right; nose pointing down on left
};
\node[mynode] (a4) at (1.5, 2.4) { \textbf{Round 5:} sad guy with backpack, legs to the left
};
\node[mynode] (a5)  at (1.5,1.4) {\textbf{Round 6:} sad guy with backpack, legs to the left
};

\node[mynode] (b0) at (-3.5,9.6) {\textbf{Round 1:} like someone dancing, with one foot on the ground and one up a little, facing left, like a head yes, like they are kind of leaning back, foot sticking out a little to the left};
\node[mynode] (b1) at (-3.8,6.6)  {\textbf{Round 2:} Like a man facing left marching with a flag over his shoulder, Face is pointy looking down
};
\node[mynode] (b2) at (-4.1,4.7) {\textbf{Round 3:} looking to the left one leg up pointy face
};
\node[mynode] (b3) at (-4.4,3.6) {\textbf{Round 4:} man marching with flag over shoulder
};
\node[mynode] (b4) at (-4.7,2.5)  { \textbf{Round 5:} marching man with flag over shoulder
};
\node[mynode] (b5)   at (-5,1.2) { \textbf{Round 6:} flag bearer};


\node[align=center, color=blue, anchor=north west] at (-10.5,10) {\textbf{Convergence} \\ \textbf{within a game}};
\node[align=center, color=div, anchor= north west] at (-5,10) {\textbf{Divergence between games}}; 

\node[] at (-10,8) {\includegraphics[width=4em]{images/tangram_A.png}};
    \begin{scope}[line width=1 pt, >=Stealth]
    %\draw [<->]   (a0.west) edge [out=230, in=150] node {.23}  (a5.west);
    %\draw [<->]   (a1.west) edge [out=210, in=150] node [left]{.13} (a5.west);
    %\draw [<->]   (a2.west) edge [bend right] node [left] {.92}(a5.west);
    %\draw [<->]   (a3.west) edge [out=210, in=90] node [left] {.47} (a5.west);
    %\draw [<->]   (a4.south) edge  node [left] {.91} (a5);
    
    \draw [<->, color=blue]   (b0) edge [out=180, in=200] node [inner sep=2ex, left, pos=.1]{.13}  (b5.south west);
    \draw [<->, color=blue]   (b1) edge [out=180, in=200] node [inner sep=3ex, left, pos=.1]{.53}  (b5.west);
    \draw [<->, color=blue]   (b2) edge [out=180, in=150] node  [inner sep=4ex, left, pos=0]{.13}  (b5.west);
    \draw [<->, color=blue]   (b3) edge [out=180, in=150] node [inner sep=3ex, left, pos=0] {.62} (b5.north west);
    \draw [<->, color=blue]   (b4.west) edge [out=270, in=90]node [inner sep=.5ex, left, pos=0]{.63} (b5.north west);
    
    \draw [<->, color=div]   (a0) edge [pos=.5, ".59"] (b0);
    \draw [<->, color=div]   (a1) edge [pos=.8, ".29"] (b1);
    \draw [<->, color=div]   (a2) edge [pos=.8, ".48"] (b2);
    \draw [<->, color=div]   (a3) edge [pos=.5, ".28"] (b3);
    \draw [<->, color=div]   (a4) edge [pos=.5, ".32"] (b4);
    \draw [<->, color=div]   (a5) edge [pos=.5, ".06"] (b5);
    

    
\end{scope}

\end{tikzpicture}
\label{sbert-diagram}
\caption{Example utterances describing the shown tangram figure produced by two 3-player games in Experiment 1. To measure convergence within a game (blue), we measured the cosine similarity between SBERT embeddings of descriptions and the embedding of the round 6 utterance (taken to be the convention). Higher cosine similarity indicates more similar meaning. To measure divergence between games (green), we measured the similarity between utterances from the same round across games.}
\end{figure}

In addition to behavioral measures, we looked at how the speaker's descriptions changed over time within and between games. We concatenated speaker's messages within a trial and used SBERT to embed the description into a high-dimensional vector space {[}TODO CITE{]}. We can compare the similarity between a pair of utterances by using the cosine similarity between their embeddings.

As a measure of convention formation, we can track how utterances describing the same tangram become increasingly similar over the course of a game. If conventions are forming we expect the similarity to the last block utterance to increase over the course of the game. If different games go in different directions with their descriptions, we'd expect this similarity between descriptions of the same image in different games to decrease over repetitions. These two metrics are illustrated with examples in Figure \ref{sbert-diagram}.

\begin{figure*}[t!]

{\centering \includegraphics[width=1\linewidth]{figs/sbert-1} 

}

\caption{Language similarity results measured with pairwise cosine similarity between embeddings of two utterances. A. Convergence of utterances within games as measured by similarity between an utterance from block 1-5 to the block 6 utterance in the same game for the same image. Dots are per-game averages, smooths are quadratic. B. Divergence of utterances across games as measured by the similarity between an utterances and utterances produced for the same image by different groups in the same block. Dots are per-image averages, smooths are quadratic.}\label{fig:sbert}
\end{figure*}

We first look at convergence, comparing utterances from the first 5 blocks of a game to the ``convention'' or last block utterance for the same figure. As visible in Figure \ref{fig:sbert}A, increasing similarity the last block utterance occurs across all conditions. In experiment 1, the similarity of the first utterance to last utterances is invariant across group size (-0.01 {[}-0.02, 0{]}), but smaller groups converge faster (-0.01 {[}-0.01, -0.01{]}). Experiment 2 shows similar patterns of utterances become more similar to the last utterance, particularly in the single speaker condition (0.09 {[}0.08, 0.09{]}) where all the utterances come from the same person, but also in the full feedback condition (0.06 {[}0.05, 0.07{]}) and to a smaller extent, in the thin condition (0.02 {[}0.01, 0.03{]}). In experiment 3, convergence is slower in thin games than thick games (-0.02 {[}-0.03, -0.02{]}) and especially thin 6 player games (-0.04 {[}-0.05, -0.02{]}).

Not only do groups reduce the lengths of their utterances, but each group is converging towards a semantic description for the figures. This is more prominent in smaller games and games with greater group coherence.

The complement to convergence within groups is divergence between groups, as different groups develop their own ways of identifying the different figures. In experiment 1, descriptions become less similar to those used to describe the same figure in other games (-0.04 {[}-0.04, -0.03{]}). Group size does not affect the cross-groups similarities in the first block (0 {[}0, 0{]}), but smaller groups diverge from each other faster than larger groups (0 {[}0, 0{]}). In experiment 2, divergence is stronger in the single speaker (-0.04 {[}-0.04, -0.04{]}) and full feedback conditions (-0.04 {[}-0.04, -0.04{]}) than in the 6 thin condition (0 {[}-0.01, 0{]}). In experiment 3, descriptions from different games get less similar over time (-0.02 {[}-0.02, -0.02{]}). There are slight differences in the initial starting points across the different conditions, as well as slight condition differences in how fast the games diverge. In particular, 6 player thin games diverge more slowly (0.02 {[}0.02, 0.02{]}).

TODO possibly say more about 6 thin and check modelling issues

Comparing embeddings of utterances also gives more details. As shown in Appendix figure, the similarity the last utterance can be seen as the cumulative similarities of consecutive utterances. Over time adjacent utterances become more similar as the utterances converge closer to a convention. As group narrow in on descriptions for each image, names for tangrams become more distinctive. While initially, many tangrams might be described in terms of shapes or body parts and directions, as the get nicknames, they differentiate more (See supp figure). All of these confirm expected patterns.

TODO say more about this

\hypertarget{todo-theres-a-note-saying-to-rerun-these-models-for-longer-with-more-extensive-mixed-effects-at-least-of-reduction-model}{%
\subsubsection{TODO there's a note saying to rerun these models for longer with more extensive mixed effects!! (at least of reduction model)}\label{todo-theres-a-note-saying-to-rerun-these-models-for-longer-with-more-extensive-mixed-effects-at-least-of-reduction-model}}

\hypertarget{general-discussion}{%
\section{General Discussion}\label{general-discussion}}

big groups / groups vary -- how do people work with these constraints and reach common understanding? study in the area of iterated reference games b/c it's well explored.

we varied several factors: group size, listener backchannel, group coherence (feedback and speaker) and measured results in terms of accuracy, reduction, and the semantic distances between utterances. Across all conditions, we saw increasing accuracy, reduction, and expected semantic patterns. People are not at chance and are mananging to come to common understandings with group mates even if the unfavorable conditions. This is impressive.

However, we also see performance gradients, by group size and interaction width. People are more accurate and faster to converge when they have fewer people to agree with, or more group coherence, or better ways to ask questions. In particular, we could interpret experiemnt 3 saying that 2 player games can cope with poor feedback, but 6 player games suffer without the thick channel.

Something about group to group variability. Creativity of descriptions, blah blah.

Intriguing results that suggest that reduction and the semantics might be less tethered -- some reduction is about getting better or common prior or something.

some paragraph about what this says, if anything, about the real world. this isn't the only group dynamic; could imagine situations where listeners can see each others work collaborate (point to each other what htey think, perhaps see feedback from speaker to one listener) which might make things reduce much faster

\begin{itemize}
\tightlist
\item
  we do have some scale; lots of overall groups, lots of talking
\end{itemize}

limitations:
* high group variability
* limited set of targets -- this sort of experimental set-up depends a lot on the stimuli being the right level of evocative or describable
* this is a high feature space -- we sampled only some pieces and grouped together b/c it's expensive and variable
* causality is also hard since there are several things that differ between 2 adn 6 player games so mechanisms are hard
* not a perfect match to real world (listeners can't copy each other) / could imagine different feedback structures
Group interactions are rich, and this experiment is necessarily a schematic simplification with a number of limitations. Real-life situations vary widely in who the interlocuters are, their relationships, their goals, and their environment (\protect\hyperlink{ref-carletta1998}{Carletta et al. 1998}, \protect\hyperlink{ref-fay2000}{Fay et al. 2000}). Our participants were a convenience sample of Prolific workers who were strangers to each other; thus we miss richness that could come from prior relationships or shared community. Reference is only one goal out of many possible communicative goals, and the tangram images are artificial.

theories?:
* it's not required that listeners talk
* once you've got first contact with a shared meaning, that's the hard part done and then it gets easier (maybe \& know you do? )

\hypertarget{methods}{%
\section{Methods}\label{methods}}

We extended the dyadic repeated reference game paradigm of Hawkins et al. (\protect\hyperlink{ref-hawkinsCharacterizingDynamicsLearning2020}{2020}) along a few dimensions. As diagrammed in Figure \ref{diagram}, three dimensions of variation we considered were group size, listener backchannel, and group coherence. In experiment 1, we expanded from dyadic reference games to group games with 2--6 players who rotated between speaker and listener roles. In experiment 2, we built on the 6 player games by exploring three variations, two which increased group coherence by increasing feedback to listeners or having a single speaker for the entire game, and one that reduced the listener backchannel. For experiment 3, we considered the extremes of group size and performance, informed by the prior experiments. The thin channel repeated the reduced-backchannel, low-group coherence condition, and we created a thick channel by combining the two sources of group coherence together. We then crossed these thick and thin condition with groups of 2 and 6 players and collected more data in each of these conditions.

For all experiments, we used Empirica (\protect\hyperlink{ref-almaatouqEmpiricaVirtualLab2020}{Almaatouq et al. 2020}) to create real-time multi-player reference games. In each game, one of the players started as the speaker who saw an array of tangrams with one highlighted (Figure \ref{game}A) and communicated which figure to click to the other players (listeners). After the speaker had identified each of the 12 images in turn, the process repeated with the same images, but a total of 6 blocks (72 trials). We recorded what participants said in the chat, as well as who selected what image and how long they took to make their selections.

These experiments were designed sequentially and pre-registered individually.\footnote{Experiment 1: \url{https://osf.io/cn9f4} for the 2-4 player groups, and \url{https://osf.io/rpz67} for the 5-6 player data run later. Experiment 2: single speaker at \url{https://osf.io/f9xyd}, full feedback at \url{https://osf.io/j5zbm}, and thin at \url{https://osf.io/k5f4t}. Experiment 3: \url{https://osf.io/untzy}} TODO more comments on pre-reg

\hypertarget{participants}{%
\subsection{Participants}\label{participants}}

\begin{table}

\caption{\label{tab:participants}The number of games in each experiment and condition. Complete games finished all 6 blocks; partial games ended early due to disconnections, but contributed at least one complete block of data. 6* indicates that some games started with fewer than 6 players or continued with fewer than 6 players after participants disconnected.}
\centering
\begin{tabular}[t]{l|l|r|r|r}
\hline
Experiment & Players & Complete & Partial & Total Participants\\
\hline
1: baseline & 2 & 15 & 4 & 38\\
\hline
1: baseline & 3 & 18 & 2 & 60\\
\hline
1: baseline & 4 & 19 & 2 & 84\\
\hline
1: baseline & 5 & 17 & 3 & 100\\
\hline
1: baseline & 6 & 12 & 6 & 108\\
\hline
2: single speaker & 6 & 15 & 3 & 108\\
\hline
2: full feedback & 6 & 13 & 4 & 102\\
\hline
2: thin & 6 & 10 & 6 & 96\\
\hline
3: thin & 2 & 35 & 3 & 76\\
\hline
3: thin & 6* & 44 & 0 & 235\\
\hline
3: thick & 2 & 39 & 3 & 84\\
\hline
3: thick & 6* & 38 & 2 & 222\\
\hline
\end{tabular}
\end{table}

Participants were recruited using the Prolific platform, and all participants self-reported as fluent native English speakers on Prolific's demographic prescreen. Participants each took part in only one experiment. Experiment 1 took place between May and July 2021, experiment 2 between March and August 2022, and experiment 3 in October 2022. As games varied in length depending on the number of participants, we paid participants based on group size, with the goal of a \$10 hourly rate. Participants were paid \$7 for 2-player games, \$8.50 for 3-player games, \$10 for 4-player games, and \$11 for 5- and 6-player games. When one player had the speaker role for the entirety of a 6-player game, they gained an addition \$2 bonus. Across all games, each participant could early up to \$2.88 in performance bonuses. A total of 1319 people participated across the 3 experiments. A breakdown of number of games and participants in each condition is shown in Table \ref{tab:participants}.

\hypertarget{materials}{%
\subsection{Materials}\label{materials}}

We used the 12 tangram images used by Hawkins et al. (\protect\hyperlink{ref-hawkinsCharacterizingDynamicsLearning2020}{2020}) and Clark \& Wilkes-Gibbs (\protect\hyperlink{ref-clarkReferringCollaborativeProcess1986}{1986}) (see Figure \ref{game}). These images were displayed in a grid with order randomized for each participant (thus descriptions such as ``top left'' were ineffective as the image might be in a different place on the speaker's and listeners' screens). The same images were used every block.

\hypertarget{procedure}{%
\subsection{Procedure}\label{procedure}}

The experimental procedure was very similar across the three experiments. We first describe the procedure used in experiment 1 and then describe the differences in later experiments.

\hypertarget{experiment-1}{%
\subsubsection{Experiment 1}\label{experiment-1}}

We implemented the experiment using Empirica, a Javascript-based platform for running real-time interactive experiments online (\protect\hyperlink{ref-almaatouqEmpiricaVirtualLab2020}{Almaatouq et al. 2020}). From Prolific, participants were directed to our website where they navigated through a self-paced series of instruction pages explaining the game. Participants had to pass a quiz to be able to play the game. They were then directed to a ``waiting room'' screen until their partners were ready.

Once the game started, participants saw screens like Figure \ref{game}A. Each trial, the speaker described the highlighted tangram image so that the listeners could identify and click it. All participants were free to use the chat box to communicate, but listeners could only click once the speaker had sent a message. Once a listener clicked, they could not change their selection. There was no signal to the speaker or other listeners about who had already made a selection.

Once all listeners had selected (or a 3-minute timer ran out), participants were given feedback (Figure \ref{game}B). Listeners learned whether they individually had chosen correctly or not; listeners who were incorrect were not told the correct answer. The speaker saw which tangram each listener had selected, but listeners did not. Listeners got 4 points for each correct answer; the speaker got points equal to the average of the listeners' points. These points translated into performance bonus at the end of the experiment.

In each block, each of the 12 tangrams was indicated to the speaker once. The same person was the speaker for an entire block, but participants rotated roles between blocks. Thus, over the course of the 6 blocks, participants were speakers 3 times in 2-player games, twice in 3-player games, once or twice in 4 and 5-player games, and once in 6-player games. Rotating the speaker was chosen to keep participants more equally engaged (the speaker role is more work), and to give a more robust test for reduction and convention.

After the game finished, participants were given a survey asking for optional demographic information and feedback on their experience with the game.

\hypertarget{differences-in-experiment-2}{%
\subsubsection{Differences in experiment 2}\label{differences-in-experiment-2}}

Experiment 2 consisted of three different variations on Experiment 1, all conducted in 6 player games. Each of these conditions differed from the experiment 1 baseline in one way. The single speaker condition differed only in that one person was designated the speaker for the entire game, rather than having the speaker role rotate. The full feedback condition differed from experiment 1 in that all participants were shown what each person had selected and what the right answer was; listeners still saw text saying whether they individually were right or wrong. This was similar to some dyadic work, such as Hawkins et al. (\protect\hyperlink{ref-hawkinsCharacterizingDynamicsLearning2020}{2020}) where listeners were shown what the right answer was during feedback. For the thin condition, we altered the chatbox interface for listeners. Instead of a textbox, listeners had 4 buttons, each of which sent a different emoji to the chat. Listeners were given suggested meanings for the 4 emojis during instructions. They could send the emojis as often as desired, for instance, initially indicating confusion, and later indicating understanding. In addition, we added notifications that appeared in the chat box saying when a player had made a selection.

\hypertarget{differences-in-experiment-3}{%
\subsubsection{Differences in experiment 3}\label{differences-in-experiment-3}}

The thin channel condition in experiment 3 was the same as the thin condition in experiment 2, above. The thick condition combined the two group coherency enhancing variations from experiment 2: one perspon was the designated speaker throughout, and the feedback participants received included the right answer and what each player had selected. TODO confirm. Across both conditions in experiment 3, notifications were sent to the chat to indicate when a participant had made a selection.

\hypertarget{data-pre-processing-and-exclusions}{%
\subsection{Data pre-processing and exclusions}\label{data-pre-processing-and-exclusions}}

Participants could use the chat box freely, which meant that the chat transcript contained some non-referential language. The first author skimmed through the chat transcripts, tagging utterances that did not refer to the current tangram. These were primarily pleasantries (``Hello''), meta-commentary about how well or fast the task was going, and confirmations or denials (``ok'', ``got it'', ``yes'', ``no''). We exclude these utterances from our analyses. Note that chat lines sometimes included non-referential words in addition to words referring to the tangrams (``ok, so it looks like a zombie'', ``yes, the one with legs''); these lines were retained intact.

In experiments 1 and 2, games did not start if there were not enough participants and ended if any participant disconnected. In experiment 3, games started after a waiting period even if they were not full and continued even after a participant disconnected (with speaker role reassigned if necessary), unless the game would drop below 2 players. The distribution of playes in these 6* player games is at TODO! The realities of online recruitment and disconnection meant that the number of games varied, although we aimed for 20 games in each condition in experiments 1 and 2, and 40 per condition in experiment 3. We excluded incomplete blocks from analyses, but included complete blocks from partial games (See Table \ref{tab:participants} for counts).

When skimming transcripts to tag non-referential utterances, we noticed that one game in the 6-player thick game had a speaker who did not give any sort of coherent descriptions, even with substantial listener prompting. We excluded this game from analyses.

\hypertarget{modelling-strategy}{%
\subsection{Modelling strategy}\label{modelling-strategy}}

TODO
In experiment 3, some of the 6 player games did not have 6 players for the entire game. We do not model this, as it is unclear at what point in the game group size is most relevant. We note that this is a conservative choice that will underestimate differences between 2 player and (genuine) 6 player games, by labelling some smaller groups as 6 player.

We ran all models in brms (CITE) with weakly regularizing priors. We were often unable to fit the full mixed effects structure that we had pre-registered in a reasonable amount of time, so we included what heirarchical effects were reasonable. (All model results and formulae are reported in TODO supplement). Accuracy results used a logistic model, other results use linear models.

\hypertarget{supplement}{%
\section{Supplement}\label{supplement}}

\begin{figure*}[t!]

{\centering \includegraphics[width=1\linewidth]{figs/listeners-1} 

}

\caption{TODO this probably goes in a supplement! }\label{fig:listeners}
\end{figure*}

\begin{figure*}[t!]

{\centering \includegraphics[width=1\linewidth]{figs/listeners2-1} 

}

\caption{TODO this probably goes in a supplement! }\label{fig:listeners2}
\end{figure*}

\begin{figure*}[t!]

{\centering \includegraphics[width=1\linewidth]{figs/other-1} 

}

\caption{Stuff probably not to include. A is similarity to first utterance. B is similarity between utterances from adjacent blocks. C is divergence in descriptions of different tangrams within a group}\label{fig:other}
\end{figure*}

\hypertarget{distinctiveness-of-tangrams}{%
\subsection{Distinctiveness of tangrams}\label{distinctiveness-of-tangrams}}

Another way of looking at how language changes over the course of the game is looking at how games start to refer to different tangrams more differently. This could reflect initial overlap in descriping many figures as sitting or standing or by leg and arm and head position.

Over the course of the game, descriptions for each tangram become more distinctive (-0.04 {[}-0.05, -0.04{]}).
In all three subexperiments, the descriptions of tangrams become more distinctive within games across time. (2a -0.05 {[}-0.05, -0.04{]}, 2b -0.03 {[}-0.03, -0.02{]}, 2c -0.03 {[}-0.03, -0.02{]}).

Tangram distinctiveness within games increased over time (-0.03 {[}-0.03, -0.03{]}). There might be more to say about other effects, but it's mostly a starting places being different in larger games and then the slopes also differ a bit?

\hypertarget{play-with-more-diagrams}{%
\subsubsection{play with more diagrams}\label{play-with-more-diagrams}}

Comparing utterances between adjacent rounds reveals similar patterns. Thin games have lower similarity between adjacent blocks (-0.12 {[}-0.16, -0.09{]}) as do larger games (-0.03 {[}-0.07, 0{]}). Later in the game adjacent blocks are more similar than earlier adjacent blocks (0.05 {[}0.04, 0.05{]}), painting an overall nonlinear convergent pattern (as seen in Figure \ref{fig:other}).

\hypertarget{accuracy-models}{%
\subsection{Accuracy models}\label{accuracy-models}}

Accuracy models were all run as logistic models with normal(0,1) priors for both betas and sd. This model was not explicitly included in the experiment 1 and 2 pre-registrations; it was included with more ambitious mixed effects (which did not run in a timely manner) in the experiment 3 pre-reg.

\begin{table}[h!]

\caption{\label{tab:unnamed-chunk-10}Experiment 1 logistic model of listener accuracy:\\ correct.num~$\sim$ block~$\times$ numPlayers~+ (1|gameId)}
\centering
\begin{tabular}[t]{lll}
\toprule
Term & Est. & CrI\\
\midrule
block & 0.44 & {}[0.31, 0.58]\\
block:numPlayers & -0.02 & {}[-0.05, 0.01]\\
Intercept & 2.10 & {}[1.57, 2.65]\\
numPlayers & -0.07 & {}[-0.2, 0.05]\\
\bottomrule
\end{tabular}
\end{table}

\begin{table}[h!]

\caption{\label{tab:unnamed-chunk-10}Experiment 2: 6 single speaker logistic model of listener accuracy:\\ correct.num~$\sim$ block~+ (1|gameId)}
\centering
\begin{tabular}[t]{lll}
\toprule
Term & Est. & CrI\\
\midrule
block & 0.45 & {}[0.39, 0.52]\\
Intercept & 1.78 & {}[1.4, 2.19]\\
\bottomrule
\end{tabular}
\end{table}

\begin{table}[h!]

\caption{\label{tab:unnamed-chunk-10}Experiment 2: 6 full feedback logistic model of listener accuracy:\\ correct.num~$\sim$ block~+ (1|gameId)}
\centering
\begin{tabular}[t]{lll}
\toprule
Term & Est. & CrI\\
\midrule
block & 0.47 & {}[0.39, 0.54]\\
Intercept & 1.35 & {}[0.59, 2.06]\\
\bottomrule
\end{tabular}
\end{table}

\begin{table}[h!]

\caption{\label{tab:unnamed-chunk-10}Experiment 2: 6 thin logistic model of listener accuracy:\\ correct.num~$\sim$ block~+ (1|gameId)}
\centering
\begin{tabular}[t]{lll}
\toprule
Term & Est. & CrI\\
\midrule
block & 0.23 & {}[0.19, 0.28]\\
Intercept & 0.88 & {}[0.64, 1.12]\\
\bottomrule
\end{tabular}
\end{table}

\begin{table}[h!]

\caption{\label{tab:unnamed-chunk-10}Experiment 3 logistic model of listener accuracy:\\ correct.num~$\sim$ block~$\times$ gameSize~$\times$ channel~+ (1|gameId)}
\centering
\begin{tabular}[t]{lll}
\toprule
Term & Est. & CrI\\
\midrule
block & 0.41 & {}[0.32, 0.5]\\
block:channelthin & -0.07 & {}[-0.18, 0.04]\\
block:gameSize6 & -0.34 & {}[-0.43, -0.25]\\
block:gameSize6:channelthin & 0.07 & {}[-0.05, 0.19]\\
channelthin & -0.36 & {}[-0.78, 0.05]\\
\addlinespace
gameSize6 & -0.64 & {}[-1.05, -0.25]\\
gameSize6:channelthin & 0.31 & {}[-0.22, 0.87]\\
Intercept & 1.69 & {}[1.39, 1.99]\\
\bottomrule
\end{tabular}
\end{table}

\pagebreak

\hypertarget{reduction-models}{%
\subsection{Reduction models}\label{reduction-models}}

Reduction models were run as linear models with an intercept prior of normal(12,20), a beta prior of normal(0,10), an sd prior of normal(0,5) and a correlation prior of lkj(1). This model was pre-registered for each experiment and run with the mixed effects structure as prespecified.

\begin{table}[h!]

\caption{\label{tab:unnamed-chunk-11}Experiment 1:\\ words~$\sim$ block~$\times$ numPlayers~+ (block|tangram)~+ (1|playerId)~+ (1|tangram:gameId)~+ (block|gameId)}
\centering
\begin{tabular}[t]{lll}
\toprule
Term & Est. & CrI\\
\midrule
block & -3.37 & {}[-4.54, -2.24]\\
block:numPlayers & -0.10 & {}[-0.36, 0.17]\\
Intercept & 16.79 & {}[11.96, 21.93]\\
numPlayers & 1.66 & {}[0.66, 2.61]\\
\bottomrule
\end{tabular}
\end{table}

\begin{table}[h!]

\caption{\label{tab:unnamed-chunk-11}Experiment 2: 6 single speaker:\\ words~$\sim$ block~+ (block|tangram)~+ (1|tangram:gameId)~+ (block|gameId)}
\centering
\begin{tabular}[t]{lll}
\toprule
Term & Est. & CrI\\
\midrule
block & -5.39 & {}[-6.46, -4.31]\\
Intercept & 29.93 & {}[24.92, 34.84]\\
\bottomrule
\end{tabular}
\end{table}

\begin{table}[h!]

\caption{\label{tab:unnamed-chunk-11}Experiment 2: 6 full feedback:\\ words~$\sim$ block~+ (block|tangram)~+ (1|tangram:gameId)~+ (block|gameId)}
\centering
\begin{tabular}[t]{lll}
\toprule
Term & Est. & CrI\\
\midrule
block & -4.68 & {}[-5.88, -3.52]\\
Intercept & 26.03 & {}[21.12, 30.58]\\
\bottomrule
\end{tabular}
\end{table}

\begin{table}[h!]

\caption{\label{tab:unnamed-chunk-11}Experiment 2: 6 thin:\\ words~$\sim$ block~+ (block|tangram)~+ (1|tangram:gameId)~+ (block|gameId)}
\centering
\begin{tabular}[t]{lll}
\toprule
Term & Est. & CrI\\
\midrule
block & -2.15 & {}[-3.44, -1.12]\\
Intercept & 20.50 & {}[17.26, 23.76]\\
\bottomrule
\end{tabular}
\end{table}

\begin{table}[h!]

\caption{\label{tab:unnamed-chunk-11}Experiment 3:\\ words~$\sim$ block~$\times$ channel~$\times$ gameSize~+ (block~$\times$ channel~$\times$ gameSize|tangram)~+ (1|tangram:gameId)~+ (block|gameId)}
\centering
\begin{tabular}[t]{lll}
\toprule
Term & Est. & CrI\\
\midrule
block & -2.29 & {}[-2.95, -1.6]\\
block:channelthin & 0.32 & {}[-0.65, 1.24]\\
block:channelthin:gameSize6 & 0.64 & {}[-0.61, 1.89]\\
block:gameSize6 & -1.21 & {}[-2.06, -0.3]\\
channelthin & 0.63 & {}[-3.18, 4.73]\\
\addlinespace
channelthin:gameSize6 & -2.11 & {}[-7.41, 2.98]\\
gameSize6 & 7.41 & {}[3.57, 11.18]\\
Intercept & 14.99 & {}[11.86, 17.89]\\
\bottomrule
\end{tabular}
\end{table}

\hypertarget{extra-reduction-model}{%
\subsubsection{Extra reduction model}\label{extra-reduction-model}}

For experiment 1, we also pre-specified models about whether the speaker's correctness (as a listener) on the prior block had an effect

Model of whether speaker's correct/incorrect answer in previous block has an effect -

words \textasciitilde{} block\emph{player\_count + block}was\_correct+ (block\textbar tangram) + (1\textbar speaker)
+ (1\textbar tangram*group)+(block\textbar group)

TODO
\pagebreak

\hypertarget{sbert-models}{%
\subsection{SBERT models}\label{sbert-models}}

For all of the models of sbert similarity, we used linear models with the priors normal(.5,.2) for intercept, normal(0,.1) for beta, and normal(0,.05) for sd.

These models were verbally described (but not formally specified) in the pre-registrations for experiment 2 in the full feedback and thin conditions and for experiment 3, for looking at divergence between games, convergence within games (compare to first, next, and last), and divergence between tangrams within games.

\hypertarget{convergence-within-games-comparison-to-last-round}{%
\subsubsection{Convergence within games: comparison to last round}\label{convergence-within-games-comparison-to-last-round}}

This is the convergence metric presented in the paper.

\begin{table}[h!]

\caption{\label{tab:unnamed-chunk-12}Experiment 1:\\ sim~$\sim$ earlier~$\times$ condition~+ (1|tangram)~+ (1|gameId)}
\centering
\begin{tabular}[t]{lll}
\toprule
Term & Est. & CrI\\
\midrule
condition & -0.01 & {}[-0.02, 0]\\
earlier & 0.09 & {}[0.08, 0.1]\\
earlier:condition & -0.01 & {}[-0.01, -0.01]\\
Intercept & 0.52 & {}[0.46, 0.57]\\
\bottomrule
\end{tabular}
\end{table}

\begin{table}[h!]

\caption{\label{tab:unnamed-chunk-12}Experiment 2: 6 single speaker:\\ sim~$\sim$ earlier~+ (1|tangram)~+ (1|gameId)}
\centering
\begin{tabular}[t]{lll}
\toprule
Term & Est. & CrI\\
\midrule
earlier & 0.09 & {}[0.08, 0.09]\\
Intercept & 0.50 & {}[0.44, 0.56]\\
\bottomrule
\end{tabular}
\end{table}

\begin{table}[h!]

\caption{\label{tab:unnamed-chunk-12}Experiment 2: 6 full feedback:\\ sim~$\sim$ earlier~+ (1|tangram)~+ (1|gameId)}
\centering
\begin{tabular}[t]{lll}
\toprule
Term & Est. & CrI\\
\midrule
earlier & 0.06 & {}[0.05, 0.07]\\
Intercept & 0.44 & {}[0.39, 0.49]\\
\bottomrule
\end{tabular}
\end{table}

\begin{table}[h!]

\caption{\label{tab:unnamed-chunk-12}Experiment 2: 6 thin:\\ sim~$\sim$ earlier~+ (1|tangram)~+ (1|gameId)}
\centering
\begin{tabular}[t]{lll}
\toprule
Term & Est. & CrI\\
\midrule
earlier & 0.02 & {}[0.01, 0.03]\\
Intercept & 0.50 & {}[0.45, 0.54]\\
\bottomrule
\end{tabular}
\end{table}

\begin{table}[h!]

\caption{\label{tab:unnamed-chunk-12}Experiment 3:\\ sim~$\sim$ earlier~$\times$ channel~$\times$ gameSize~+ (1|tangram)~+ (1|gameId)}
\centering
\begin{tabular}[t]{lll}
\toprule
Term & Est. & CrI\\
\midrule
channelthin & -0.03 & {}[-0.08, 0.01]\\
channelthin:gameSize6 & 0.04 & {}[-0.02, 0.1]\\
earlier & 0.08 & {}[0.07, 0.09]\\
earlier:channelthin & -0.02 & {}[-0.03, -0.02]\\
earlier:channelthin:gameSize6 & -0.04 & {}[-0.05, -0.02]\\
\addlinespace
earlier:gameSize6 & 0.01 & {}[0, 0.02]\\
gameSize6 & -0.07 & {}[-0.11, -0.02]\\
Intercept & 0.58 & {}[0.54, 0.62]\\
\bottomrule
\end{tabular}
\end{table}

\pagebreak

\hypertarget{divergence-across-games}{%
\subsubsection{Divergence across games}\label{divergence-across-games}}

To look at how games diverged from each other \ldots{} TODO

\begin{table}[h!]

\caption{\label{tab:unnamed-chunk-13}Experiment 1:\\ sim~$\sim$ block~$\times$ condition~+ (1|tangram)}
\centering
\begin{tabular}[t]{lll}
\toprule
Term & Est. & CrI\\
\midrule
block & -0.04 & {}[-0.04, -0.03]\\
block:condition & 0.00 & {}[0, 0]\\
condition & 0.00 & {}[0, 0]\\
Intercept & 0.47 & {}[0.43, 0.51]\\
\bottomrule
\end{tabular}
\end{table}

\begin{table}[h!]

\caption{\label{tab:unnamed-chunk-13}Experiment 2: 6 single speaker:\\ sim~$\sim$ block~+ (1|tangram)}
\centering
\begin{tabular}[t]{lll}
\toprule
Term & Est. & CrI\\
\midrule
block & -0.04 & {}[-0.04, -0.04]\\
Intercept & 0.48 & {}[0.44, 0.53]\\
\bottomrule
\end{tabular}
\end{table}

\begin{table}[h!]

\caption{\label{tab:unnamed-chunk-13}Experiment 2: 6 full feedback:\\ sim~$\sim$ block~+ (1|tangram)}
\centering
\begin{tabular}[t]{lll}
\toprule
Term & Est. & CrI\\
\midrule
block & -0.04 & {}[-0.04, -0.04]\\
Intercept & 0.50 & {}[0.46, 0.55]\\
\bottomrule
\end{tabular}
\end{table}

\begin{table}[h!]

\caption{\label{tab:unnamed-chunk-13}Experiment 2: 6 thin:\\ sim~$\sim$ block~+ (1|tangram)}
\centering
\begin{tabular}[t]{lll}
\toprule
Term & Est. & CrI\\
\midrule
block & 0.00 & {}[-0.01, 0]\\
Intercept & 0.43 & {}[0.41, 0.46]\\
\bottomrule
\end{tabular}
\end{table}

\begin{table}[h!]

\caption{\label{tab:unnamed-chunk-13}Experiment 3:\\ sim~$\sim$ block~$\times$ channel~$\times$ gameSize~+ (1|tangram)}
\centering
\begin{tabular}[t]{lll}
\toprule
Term & Est. & CrI\\
\midrule
block & -0.02 & {}[-0.02, -0.02]\\
block:channelthin & 0.00 & {}[0, 0]\\
block:channelthin:gameSize6 & 0.02 & {}[0.02, 0.02]\\
block:gameSize6 & -0.01 & {}[-0.01, -0.01]\\
channelthin & 0.01 & {}[0.01, 0.02]\\
\addlinespace
channelthin:gameSize6 & -0.03 & {}[-0.04, -0.02]\\
gameSize6 & 0.05 & {}[0.05, 0.05]\\
Intercept & 0.41 & {}[0.37, 0.45]\\
\bottomrule
\end{tabular}
\end{table}

\pagebreak

\hypertarget{divergence-across-tangrams}{%
\subsubsection{Divergence across tangrams}\label{divergence-across-tangrams}}

\begin{table}[h!]

\caption{\label{tab:unnamed-chunk-14}Experiment 1:\\ sim~$\sim$ block~$\times$ condition~+ (1|gameId)}
\centering
\begin{tabular}[t]{lll}
\toprule
Term & Est. & CrI\\
\midrule
block & -0.04 & {}[-0.05, -0.04]\\
block:condition & 0.00 & {}[0, 0]\\
condition & 0.00 & {}[-0.01, 0.01]\\
Intercept & 0.43 & {}[0.38, 0.47]\\
\bottomrule
\end{tabular}
\end{table}

\begin{table}[h!]

\caption{\label{tab:unnamed-chunk-14}Experiment 2: 6 single speaker:\\ sim~$\sim$ block~+ (1|gameId)}
\centering
\begin{tabular}[t]{lll}
\toprule
Term & Est. & CrI\\
\midrule
block & -0.05 & {}[-0.05, -0.04]\\
Intercept & 0.42 & {}[0.39, 0.44]\\
\bottomrule
\end{tabular}
\end{table}

\begin{table}[h!]

\caption{\label{tab:unnamed-chunk-14}Experiment 2: 6 full feedback:\\ sim~$\sim$ block~+ (1|gameId)}
\centering
\begin{tabular}[t]{lll}
\toprule
Term & Est. & CrI\\
\midrule
block & -0.05 & {}[-0.05, -0.04]\\
Intercept & 0.46 & {}[0.42, 0.5]\\
\bottomrule
\end{tabular}
\end{table}

\begin{table}[h!]

\caption{\label{tab:unnamed-chunk-14}Experiment 2: 6 thin:\\ sim~$\sim$ block~+ (1|gameId)}
\centering
\begin{tabular}[t]{lll}
\toprule
Term & Est. & CrI\\
\midrule
block & -0.03 & {}[-0.03, -0.02]\\
Intercept & 0.43 & {}[0.39, 0.47]\\
\bottomrule
\end{tabular}
\end{table}

\begin{table}[h!]

\caption{\label{tab:unnamed-chunk-14}Experiment 3:\\ sim~$\sim$ block~$\times$ channel~$\times$ gameSize~+ (1|gameId)}
\centering
\begin{tabular}[t]{lll}
\toprule
Term & Est. & CrI\\
\midrule
block & -0.03 & {}[-0.03, -0.03]\\
block:channelthin & 0.00 & {}[0, 0]\\
block:channelthin:gameSize6 & 0.01 & {}[0.01, 0.01]\\
block:gameSize6 & -0.01 & {}[-0.01, -0.01]\\
channelthin & 0.04 & {}[0, 0.08]\\
\addlinespace
channelthin:gameSize6 & -0.05 & {}[-0.1, -0.01]\\
gameSize6 & 0.08 & {}[0.04, 0.12]\\
Intercept & 0.38 & {}[0.35, 0.4]\\
\bottomrule
\end{tabular}
\end{table}

\pagebreak

\hypertarget{convergence-to-next}{%
\subsubsection{convergence to next}\label{convergence-to-next}}

We also looked at how similar an utterance was to the next round utterance: this can be thought of as the derivative of the to-last comparison. (although cosine similarities are not actually additive in the same way integrals are)

\begin{table}[h!]

\caption{\label{tab:unnamed-chunk-15}Experiment 1:\\ sim~$\sim$ earlier~$\times$ condition~+ (1|tangram)~+ (1|gameId)}
\centering
\begin{tabular}[t]{lll}
\toprule
Term & Est. & CrI\\
\midrule
condition & 0.00 & {}[-0.01, 0.01]\\
earlier & 0.06 & {}[0.05, 0.07]\\
earlier:condition & -0.01 & {}[-0.01, -0.01]\\
Intercept & 0.59 & {}[0.54, 0.64]\\
\bottomrule
\end{tabular}
\end{table}

\begin{table}[h!]

\caption{\label{tab:unnamed-chunk-15}Experiment 2: 6 single speaker:\\ sim~$\sim$ earlier~+ (1|tangram)~+ (1|gameId)}
\centering
\begin{tabular}[t]{lll}
\toprule
Term & Est. & CrI\\
\midrule
earlier & 0.04 & {}[0.04, 0.05]\\
Intercept & 0.66 & {}[0.62, 0.7]\\
\bottomrule
\end{tabular}
\end{table}

\begin{table}[h!]

\caption{\label{tab:unnamed-chunk-15}Experiment 2: 6 full feedback:\\ sim~$\sim$ earlier~+ (1|tangram)~+ (1|gameId)}
\centering
\begin{tabular}[t]{lll}
\toprule
Term & Est. & CrI\\
\midrule
earlier & 0.02 & {}[0.01, 0.02]\\
Intercept & 0.61 & {}[0.57, 0.64]\\
\bottomrule
\end{tabular}
\end{table}

\begin{table}[h!]

\caption{\label{tab:unnamed-chunk-15}Experiment 2: 6 thin:\\ sim~$\sim$ earlier~+ (1|tangram)~+ (1|gameId)}
\centering
\begin{tabular}[t]{lll}
\toprule
Term & Est. & CrI\\
\midrule
earlier & 0.01 & {}[0, 0.02]\\
Intercept & 0.53 & {}[0.49, 0.58]\\
\bottomrule
\end{tabular}
\end{table}

\begin{table}[h!]

\caption{\label{tab:unnamed-chunk-15}Experiment 3:\\ sim~$\sim$ earlier~$\times$ channel~$\times$ gameSize~+ (1|tangram)~+ (1|gameId)}
\centering
\begin{tabular}[t]{lll}
\toprule
Term & Est. & CrI\\
\midrule
channelthin & -0.12 & {}[-0.16, -0.09]\\
channelthin:gameSize6 & 0.00 & {}[-0.05, 0.05]\\
earlier & 0.05 & {}[0.04, 0.05]\\
earlier:channelthin & -0.01 & {}[-0.02, 0]\\
earlier:channelthin:gameSize6 & -0.02 & {}[-0.03, -0.01]\\
\addlinespace
earlier:gameSize6 & 0.00 & {}[-0.01, 0]\\
gameSize6 & -0.03 & {}[-0.07, 0]\\
Intercept & 0.71 & {}[0.68, 0.75]\\
\bottomrule
\end{tabular}
\end{table}

\pagebreak

\hypertarget{divergence-from-first}{%
\subsubsection{divergence from first}\label{divergence-from-first}}

We also looked at how similar an utterance was to the first round utterance. This is not very informative because first round utterances tend to be pretty unwieldy. TODO explain more or don't include

\begin{table}[h!]

\caption{\label{tab:unnamed-chunk-16}Experiment 1:\\ sim~$\sim$ later~$\times$ condition~+ (1|tangram)~+ (1|gameId)}
\centering
\begin{tabular}[t]{lll}
\toprule
Term & Est. & CrI\\
\midrule
condition & -0.01 & {}[-0.02, 0]\\
Intercept & 0.65 & {}[0.59, 0.7]\\
later & -0.03 & {}[-0.04, -0.02]\\
later:condition & 0.00 & {}[0, 0]\\
\bottomrule
\end{tabular}
\end{table}

\begin{table}[h!]

\caption{\label{tab:unnamed-chunk-16}Experiment 2: 6 single speaker:\\ sim~$\sim$ later~+ (1|tangram)~+ (1|gameId)}
\centering
\begin{tabular}[t]{lll}
\toprule
Term & Est. & CrI\\
\midrule
Intercept & 0.68 & {}[0.63, 0.73]\\
later & -0.04 & {}[-0.05, -0.03]\\
\bottomrule
\end{tabular}
\end{table}

\begin{table}[h!]

\caption{\label{tab:unnamed-chunk-16}Experiment 2: 6 full feedback:\\ sim~$\sim$ later~+ (1|tangram)~+ (1|gameId)}
\centering
\begin{tabular}[t]{lll}
\toprule
Term & Est. & CrI\\
\midrule
Intercept & 0.64 & {}[0.58, 0.71]\\
later & -0.04 & {}[-0.05, -0.04]\\
\bottomrule
\end{tabular}
\end{table}

\begin{table}[h!]

\caption{\label{tab:unnamed-chunk-16}Experiment 2: 6 thin:\\ sim~$\sim$ later~+ (1|tangram)~+ (1|gameId)}
\centering
\begin{tabular}[t]{lll}
\toprule
Term & Est. & CrI\\
\midrule
Intercept & 0.54 & {}[0.49, 0.58]\\
later & -0.01 & {}[-0.02, 0]\\
\bottomrule
\end{tabular}
\end{table}

\begin{table}[h!]

\caption{\label{tab:unnamed-chunk-16}Experiment 3:\\ sim~$\sim$ later~$\times$ channel~$\times$ gameSize~+ (1|tangram)~+ (1|gameId)}
\centering
\begin{tabular}[t]{lll}
\toprule
Term & Est. & CrI\\
\midrule
channelthin & -0.08 & {}[-0.12, -0.03]\\
channelthin:gameSize6 & -0.06 & {}[-0.13, 0]\\
gameSize6 & -0.02 & {}[-0.06, 0.03]\\
Intercept & 0.72 & {}[0.68, 0.76]\\
later & -0.03 & {}[-0.04, -0.03]\\
\addlinespace
later:channelthin & 0.01 & {}[0, 0.02]\\
later:channelthin:gameSize6 & 0.02 & {}[0.01, 0.03]\\
later:gameSize6 & -0.01 & {}[-0.02, 0]\\
\bottomrule
\end{tabular}
\end{table}

\hypertarget{extra-emoji-analysis}{%
\subsubsection{Extra emoji analysis}\label{extra-emoji-analysis}}

Written about 6thin in experiment 2 and for 2 and 6 thin in 3
Additionally, exclusive to this condition, we will analyse the distribution of emoji's produced as a function of block and its relation to accuracy and speaker utterance length.

\hypertarget{references}{%
\section{References}\label{references}}

\setlength{\parindent}{-0.1in} 
\setlength{\leftskip}{0.125in}

\noindent

\hypertarget{refs}{}
\begin{CSLReferences}{1}{0}
\leavevmode\vadjust pre{\hypertarget{ref-almaatouqEmpiricaVirtualLab2020}{}}%
Almaatouq A, Becker J, Houghton JP, Paton N, Watts DJ, Whiting ME (2020) \href{http://arxiv.org/abs/2006.11398}{Empirica: A virtual lab for high-throughput macro-level experiments}. \emph{ArXiv200611398 Cs}

\leavevmode\vadjust pre{\hypertarget{ref-branigan2006}{}}%
Branigan H (2006) Perspectives on multi-party dialogue. \emph{Research on Language and Computation} \textbf{4}:153--177

\leavevmode\vadjust pre{\hypertarget{ref-carletta1998}{}}%
Carletta J, Garrod S, Fraser-Krauss H (1998) Placement of {Authority} and {Communication Patterns} in {Workplace Groups}: {The Consequences} for {Innovation}. \emph{Small Group Research} \textbf{29}:531--559. doi:\href{https://doi.org/10.1177/1046496498295001}{10.1177/1046496498295001}

\leavevmode\vadjust pre{\hypertarget{ref-clarkReferringCollaborativeProcess1986}{}}%
Clark HH, Wilkes-Gibbs D (1986) \href{http://www.speech.kth.se/~edlund/bielefeld/references/clark-and-wilkes-gibbs-1986.pdf}{Referring as a collaborative process}. \emph{Cognition}

\leavevmode\vadjust pre{\hypertarget{ref-fay2000}{}}%
Fay N, Garrod S, Carletta J (2000) Group {Discussion} as {Interactive Dialogue} or as {Serial Monologue}: {The Influence} of {Group Size}. \emph{Psychol Sci} \textbf{11}:481--486. doi:\href{https://doi.org/10.1111/1467-9280.00292}{10.1111/1467-9280.00292}

\leavevmode\vadjust pre{\hypertarget{ref-fox-tree2013}{}}%
Fox Tree JE, Clark NB (2013) Communicative {Effectiveness} of {Written Versus Spoken Feedback}. \emph{Discourse Processes} \textbf{50}:339--359. doi:\href{https://doi.org/10.1080/0163853X.2013.797241}{10.1080/0163853X.2013.797241}

\leavevmode\vadjust pre{\hypertarget{ref-ginzburg2005}{}}%
Ginzburg J, Fernandez R (2005) Action at a distance: The difference between dialogue and multilogue. \emph{Proceedings of DIALOR}:9

\leavevmode\vadjust pre{\hypertarget{ref-haber2019}{}}%
Haber J, Baumgärtner T, Takmaz E, Gelderloos L, Bruni E, Fernández R (2019) The {PhotoBook Dataset}: {Building Common Ground} through {Visually-Grounded Dialogue}. In: \emph{Proc. 57th {Annu}. {Meet}. {Assoc}. {Comput}. {Linguist}.} {Association for Computational Linguistics}, {Florence, Italy}, p 1895--1910. Available from: \url{https://www.aclweb.org/anthology/P19-1184} {[}Last accessed 1 February 2022{]}. doi:\href{https://doi.org/10.18653/v1/P19-1184}{10.18653/v1/P19-1184}

\leavevmode\vadjust pre{\hypertarget{ref-hawkinsCharacterizingDynamicsLearning2020}{}}%
Hawkins RD, Frank MC, Goodman ND (2020) \href{http://arxiv.org/abs/1912.07199}{Characterizing the dynamics of learning in repeated reference games}. \emph{ArXiv191207199 Cs}

\leavevmode\vadjust pre{\hypertarget{ref-horton2002}{}}%
Horton WS, Gerrig RJ (2002) {SpeakersÕ} experiences and audience design: Knowing when and knowing how to adjust utterances to addresseesq. \emph{Journal of Memory and Language}:18

\leavevmode\vadjust pre{\hypertarget{ref-horton2005}{}}%
Horton WS, Gerrig RJ (2005) The impact of memory demands on audience design during language production. \emph{Cognition} \textbf{96}:127--142. doi:\href{https://doi.org/10.1016/j.cognition.2004.07.001}{10.1016/j.cognition.2004.07.001}

\leavevmode\vadjust pre{\hypertarget{ref-kraussChangesReferencePhrases1964}{}}%
Krauss RM, Weinheimer S (1964) Changes in reference phrases as a function of frequency of usage in social interaction: A preliminary study. \emph{Psychon Sci} \textbf{1}:113--114. doi:\href{https://doi.org/10.3758/BF03342817}{10.3758/BF03342817}

\leavevmode\vadjust pre{\hypertarget{ref-kraussConcurrentFeedbackConfirmation1966}{}}%
Krauss RM, Weinheimer S (1966) Concurrent feedback, confirmation, and the encoding of referents in verbal communication. \emph{Journal of Personality and Social Psychology} \textbf{4}:343--346. doi:\href{https://doi.org/10.1037/h0023705}{10.1037/h0023705}

\leavevmode\vadjust pre{\hypertarget{ref-schober1989}{}}%
Schober MF, Clark HH (1989) Understanding by addressees and overhearers. \emph{Cognitive Psychology} \textbf{21}:211--232. doi:\href{https://doi.org/10.1016/0010-0285(89)90008-X}{10.1016/0010-0285(89)90008-X}

\leavevmode\vadjust pre{\hypertarget{ref-tolins2016}{}}%
Tolins J, Fox Tree JE (2016) Overhearers {Use Addressee Backchannels} in {Dialog Comprehension}. \emph{Cogn Sci} \textbf{40}:1412--1434. doi:\href{https://doi.org/10.1111/cogs.12278}{10.1111/cogs.12278}

\leavevmode\vadjust pre{\hypertarget{ref-traum2004}{}}%
Traum D (2004) Issues in {Multiparty Dialogues}. In: Dignum F (ed) \emph{Advances in {Agent Communication}}. {Springer Berlin Heidelberg}, {Berlin, Heidelberg}, p 201--211. Available from: \url{http://link.springer.com/10.1007/978-3-540-24608-4_12} {[}Last accessed 1 February 2022{]}. doi:\href{https://doi.org/10.1007/978-3-540-24608-4_12}{10.1007/978-3-540-24608-4\_12}

\end{CSLReferences}


\end{document}
